% !TEX root = diplom.tex
\documentclass[14pt,a4paper]{extarticle}

% ======== Кодировка и кириллица ========
\usepackage[T2A]{fontenc}
\usepackage[utf8]{inputenc}
\usepackage[russian]{babel}

% ======== Шрифт и форматирование по СТП 20-04-2008 ========
\usepackage{geometry}
\geometry{a4paper, left=30mm, right=15mm, top=20mm, bottom=20mm}
\usepackage{setspace}
\onehalfspacing


\usepackage{mathptmx} % Использование шрифта Times New Roman
\renewcommand{\rmdefault}{ptm}
\renewcommand{\sfdefault}{phv}

% ======== Заголовки и отступы ========
\usepackage{titlesec}
\titleformat{\section}{\bfseries\normalsize}{\thesection.}{1em}{}
\titleformat{\subsection}{\normalsize\bfseries}{\thesubsection.}{1em}{}
\setlength{\parindent}{1.25cm} % Отступы
\setlength{\parskip}{0pt} % Без интервалов между абзацами
\usepackage{indentfirst}

\begin{document}

% ======== Титульный лист ========
\begin{titlepage}
    \centering
    \large
    Министерство и науки Российской Федерации\\
    \textbf{Федеральное государственное бюджетное образовательное учреждение высшего образования}\\
    \textbf{«Университет информационных технологий»}\\[2cm]

    \textbf{\Large ДИПЛОМНАЯ РАБОТА}\\[0.5cm]
    на тему:\\[0.3cm]
    \textbf{<<Система менеджмента качества на РУП «Минский тракторный завод» и направления её совершенствования>>}\\[4cm]

    Студент:\\
    ФМ, 5-й курс, ДКП-1\\
    (подпись) В.Л. Громов\\[1cm]

    Руководитель:\\
    кан. экон. наук, доцент\\
    (подпись) М.М. Лесовой\\[1cm]

    Консультанты:\\
    (подпись) А.А. Романов\\
    (подпись) В.Б. Симонов\\[1cm]

    Нормоконтролёр:\\
    (подпись) В.В. Базанов\\[3cm]

    Минск 2003
\end{titlepage}

\tableofcontents
\newpage

% ======== Аннотация ========
\section*{Аннотация}
Дипломная работа посвящена разработке интеллектуальной системы анализа текстов на естественном языке.
В работе рассматриваются методы машинного обучения и обработки естественного языка (NLP) для автоматической классификации и извлечения смысловых единиц из текстовых данных.
Разработана программная реализация на языке Python с использованием библиотеки \texttt{scikit-learn}.
Практическая часть включает анализ русскоязычных текстов и построение модели классификации по тематике.

\newpage
\section*{Введение}
Современные технологии обработки текстовой информации играют ключевую роль в развитии искусственного интеллекта.
В связи с ростом объёмов текстовых данных возрастает необходимость в автоматизированных системах, способных анализировать и понимать естественный язык.
Цель данной работы — исследование и разработка интеллектуальной системы анализа текстов, обеспечивающей классификацию и тематическую группировку документов.

Для достижения цели были поставлены следующие задачи:
\begin{itemize}
    \item анализ существующих методов обработки естественного языка;
    \item проектирование архитектуры системы;
    \item реализация алгоритмов машинного обучения для классификации текстов;
    \item экспериментальная оценка качества работы модели.
\end{itemize}

\newpage
\section{Аналитический обзор}
В последние годы значительное развитие получили алгоритмы глубокого обучения, применяемые для обработки естественного языка.
Модели семейства трансформеров, такие как BERT, GPT и RoBERTa, демонстрируют высокие результаты в задачах анализа тональности, классификации и генерации текста.

Однако для русскоязычных данных до сих пор актуальны классические подходы, основанные на векторизации слов (TF-IDF, Word2Vec) и методах машинного обучения, таких как логистическая регрессия и SVM.
Комбинирование современных и традиционных подходов позволяет достичь баланса между точностью и вычислительной эффективностью.

\subsection{Сравнение подходов}
Методы глубокого обучения требуют больших вычислительных ресурсов и больших объёмов размеченных данных.
Классические модели менее требовательны, но иногда уступают по качеству.
В данной работе выбран гибридный подход — использование TF-IDF для векторизации и логистической регрессии как классификатора.

\newpage
\section{Проектирование и реализация системы}
Разрабатываемая система включает три основных модуля:
\begin{enumerate}
    \item модуль предобработки текстов;
    \item модуль обучения модели классификации;
    \item модуль анализа новых данных.
\end{enumerate}

В качестве языка программирования выбран Python, а для реализации модели — библиотека \texttt{scikit-learn}.
Архитектура
системы предусматривает возможность расширения функционала за счёт подключения дополнительных алгоритмов и источников данных.

\subsection{Описание работы программы}
Программа принимает на вход текстовые документы, выполняет их токенизацию, удаление стоп-слов, лемматизацию и векторизацию.
Затем обученная модель классификации присваивает каждому документу тематическую метку.
Результаты анализа визуализируются в виде таблицы или диаграммы.

\newpage
\section*{Заключение}
В результате выполнения дипломной работы была разработана интеллектуальная система анализа текстов на естественном языке.
Проведён анализ существующих подходов, реализована программная часть и выполнено тестирование на реальных данных.
Полученные результаты подтверждают эффективность предложенного решения.

Дальнейшее развитие проекта связано с интеграцией методов глубокого обучения и расширением функционала системы.

\end{document}

